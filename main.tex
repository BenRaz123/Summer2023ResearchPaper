\documentclass{article}
\usepackage{biblatex}[backend=biber,style=plain]
\addbibresource{main.bib}
\title{\bfseries{
    Examination of the Effects of Programming Languages on Developer Productivity, as Measured by Time
}}
\author{Ben Raz \texttt{<ben.raz@student.winstonprep.edu>}}
\date{\today}

\begin{document}

\maketitle

\begin{abstract}
    In this study, we test whether programming languages have an effect on developer productivity and how that effect manifests itself by implementing a real-world algorithm in each language. We find that lightweight scripting languages like Python and JavaScript give developers the most productivity while low-level languages like Rust and especially C++ do more to hinder developer productivity. Additionally, we summarize the current literature and discuss the limitations of this study.
\end{abstract}

\section{Introduction}

In one year on earth, developers will spend around 7 billion hours\cite{prechelt2000} interacting with a programming language. That is comparable to the lifetime of planets. At this scale, productivity clearly matters. A change of as little as 1\% can free up tens of millions of hours. These hours can go towards fueling innovation. Because of the importance of programmer productivity, we study which popular programming language makes developers the most productive. Additionally, we analyze the tradeoffs of each language.

\printbibliography


\end{document}
